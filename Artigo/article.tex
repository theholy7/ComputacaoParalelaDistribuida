\documentclass[a4paper,prd,twocolumn,nofootinbib,superscriptaddress,floatfix]{revtex4}
%\documentclass[prd,twocolumn,nofootinbib,showpacs]{revtex4-1}

\usepackage{cancel}
\usepackage{graphicx}
\usepackage{epsfig}
\usepackage{amsmath,amsfonts,amssymb}
\usepackage[latin9]{inputenc}
\usepackage[portuguese]{babel} 

\hyphenation{}
\begin{document}

%\linenumbers

\title{Lowest Common Subsequence - Serial and Parallel Implementation}

\author{Jos� Antunes} 
\author{C�sar Alves}
\author{Mauro Machado}

\affiliation{Departamento de F�sica, Instituto Superior T�cnico, Universidade de Lisboa, Lisboa, Portugal}

\begin{abstract}
ALGORITMO IMPLEMENTADO - SERIAL - PARALLEL - BEST SPEED SERIAL - BEST SPEED PARALLEL - SPEED UP
\end{abstract}

\maketitle

\section{Introduction}

\section{Methods}
\subsection{Serial Implementation}
%\begin{figure}[h!]
%  \centering
%      \includegraphics[width=0.5\textwidth]{esquema.png}
%  \caption{Circuito utilizado na constru��o da coluna~\cite{esquema}.}
%   \label{fig:2}
%\end{figure}

\subsection{Parallel Implementation}


\section{Results}


\section{Discussion}


\section{Conclusions}

\vspace{-2mm}
\section{Agradecimentos}
Gostaria de agradecer ao NFIST - N�cleo de F�sica do IST - pelo apoio na obten��o de material para a execu��o deste trabalho e de agradecer ao R�ben Marques e ao Lino Marques pelo conhecimento e talento na fabrica��o de pe�as necess�rias � coluna. Sem este esfor�o de equipa, este trabalho n�o poderia ser realizado t�o metodicamente como desejado.

Gostaria tamb�m de agradecer ao Professor Filipe Joaquim pelos conselhos dados nas aulas sobre a escrita de um artigo e apresenta��o de um trabalho cient�fico. Gra�as ao tempo que disponibilizou para nos ensinar nas aulas foi mais f�cil organizar as ideias e planear o trabalho, tornando-me mais eficiente a escrev�-lo.


\begin{thebibliography}{99}

\bibitem{esquema}
K. Cantrell, "A Study of the Plasma Tweeter", B.Sc. Thesis, Ball State University (2011)

\bibitem{ideia}
M. Hopkins and T. Houlhan, "The Plasma Speaker: Construction and Characterization of both Full-bridge and Single-ended driving circuits", Project Report, University of Illinois at Urbana-Champaign (2012)

\bibitem{ideiageral}
D. Severinsen and G. Sen Gupta, "Design and Evaluation of Electronic Circuit for Plasma Speaker", Proceedings of the World Congress on Engineering 2013 Vol II (2013)

\bibitem{arstuff}
L. Wayne Sieck, John T. Herron, and David S. Green, Plasma Chem., Plasma P., Vol. 20, No. 2, 2000
\bibitem{arstuff2}
John T. Herron and David S. Green, Plasma Chem., Plasma P., Vol. 21, No. 3, 2001
\bibitem{arstuff3}
K.H. Becker, U. Kogelschatz, K.H. Schoenbach, R.J. Barker, "Non-Equilibrium Air Plasmas at Atmospheric Pressure", p. 130, Institute of Physics Publishing, Bristol, UK (2005)
\bibitem{Mathematica}
Wolfram Research, Inc., Mathematica, Version 9.0, Champaign, IL (2012).
\end{thebibliography}

\end{document}
